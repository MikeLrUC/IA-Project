\documentclass[journal, a4paper]{IEEEtran}

\usepackage[left=2cm,right=1.5cm,top=1.5cm,bottom=1.5cm]{geometry}
\usepackage{graphicx} 
\usepackage[utf8]{inputenc}
\usepackage[english]{babel}
\usepackage{csquotes}
\usepackage[parfill]{parskip}
\usepackage{parskip}
\usepackage{url}    
\usepackage{hyperref}
\usepackage{authblk}

\hypersetup{
  colorlinks = true,
  linkcolor = blue,
  filecolor = magenta,      
  urlcolor = cyan,
  pdftitle = {Pedestrian Semaphore Classification - An Active Learning Approach}
  pdfauthor = {Miguel Rabuge, Gabriel Fernandes, Pedro Rodrigues},
  pdfsubject = {Final Paper},
  pdfkeywords = {Artificial Intelligence, Machine Learning, Active Learning, 
  Autonomous Pedestrian Module, Image Classification},
  pdfproducer = {Latex with hyperref},
  pdfcreator = {pdflatex},
}

\begin{document}

\title{Pedestrian Semaphore Classification \\ An Active Learning Approach}
\author[1]{Gabriel Fernandes}
\author[2]{Miguel Rabuge}
\author[3]{Pedro Rodrigues}

\affil[1,2,3]{Department of Informatics Engineering, University of Coimbra, Portugal}
{
  \makeatletter
  \renewcommand\AB@affilsepx{: \protect\Affilfont}
  \makeatother

  \affil[1]{gabrielf@student.dei.uc.pt}
  \affil[2]{rabuge@student.dei.uc.pt}
  \affil[3]{pedror@stundent.dei.uc.pt}
}

\maketitle

\begin{abstract}
The environments, which pedestrians with some form of disability circulate are 
often extremely risky, more so than for "healthy" pedestrians. Not only the 
environments are dynamic, and full of uncertainty but, since an impaired 
individual is not able to record quality sensory information his ability to 
navigate through the environment, without endangering himself, is compromised.
The most common solution for this problem is some form of personal companion, 
e.g “a guide dog”, that helps the individual circulate. Training these kinds of 
companions is often slow and expensive... We propose an “Active Learning”(AL) 
approach for training a classifier that could be integrated in an artificial 
personal companion module (...)
\end{abstract}

\begin{keywords}
Artificial Intelligence, Machine Learning, Active Learning, Autonomous Pedestrian Module, 
Image Classification.
\end{keywords}

\section{Introduction}
Nowadays, there is a lot of effort being put into solving the problem of 
autonomous driving, not only because of research but also because of the security 
wand comfort that autonomous vehicles may bring to our lives. Despite the fact of 
this problem being hard, it is made easy because of the environment
where the action of driving happens. There is a set of rules about how to drive, 
and where to drive to avoid potential accidents resulting in collisions with 
other vehicles, persons or objects. In other words, the environment is dynamic 
and risky, but it is less chaotic and more well-defined than others, which becomes 
more suited for the development of artificial intelligence approaches for training 
vehicles to be autonomous. 

To our knowledge, there has not been a lot of development of similar techniques 
for pedestrians, namely the ones that show some form of disability that could put 
them in danger if not accompanied by some form of personal companion. The 
pedestrian environment, unlike the one where vehicles circulate, does not have a 
set of rules describing the way people should walk nor a place where the action of 
walking takes place. The pedestrian may be forced to walk on a sidewalk, a 
(lighted) crosswalk, in rugged terrain or even in the car lane if no sidewalk 
exists. Additionally, the objects that a pedestrian needs to avoid obviously 
increases since a person needs not to worry only about walking itself, but also 
other obstacles with more unpredictable moving patterns. Therefore, the 
environment is often so chaotic that travelling by foot requires assistance and 
makes the problem less approachable from an artificial intelligence standpoint.

Since the problem of autonomous mobility for disabled pedestrians is not 
particularly developed, and the only way such mobility can be achieved is with 
the help of some form of companion, (like for instance a guide dog which requires 
a lot of training to successfully guide people), our contribution is to advance 
inside this field in terms of pedestrian semaphore detection and classification 
for a crosswalk-taking decision. The choice of an active learning (AL) approach 
is justified by the lack of datasets for this specific task. Therefore, AL is the 
perfect methodology to apply for this niche, but yet significant, application, 
since it can extract better accuracy from the models with fewer data.

In short, with this project we aim to explore an alternative methodology for 
this problem, by comparing the State-of-the-Art methods performance with our own, 
using various metrics. Our main objective is to build a classifier that should be 
fast to classify and highly accurate, without the need for large amounts of data.

\section{State Of The Art}

In the past few years, the State of the Art of image classifiers tends to revolve 
around Convolutional Neural Networks (CNN), due to the large amounts of data and 
the processing power currently available, which have turned out to be highly 
rewarding when both conditions are true.

In terms of pedestrian assistive agents, there has not been a lot of research 
related to it since a ``Pedestrian Assistant'' is a relatively recent topic. 
The most recent scientific paper that we found for it, uses a standard 
State-of-the-Art CNN for the pedestrian traffic lights detection and 
classification module of their project, trained by using a newly introduced 
dataset “Pedestrian and Vehicle Traffic Lights” (PVTL), which is a pre-processed 
subset of Mapillary’s vast image dataset.

Regarding Active Learning, this approach aims to iteratively find the most 
informative data points to train the model, through the use of entropy, etc. 
This methodology reveals highly rewarding, generating good models with low 
data usage.

\section{Materials: Datasets and Frameworks}

In terms of data sets we found only one good resource
online for our objective:
\begin{itemize}
  \item ``Pedestrian and Vehicle Traffic Lights''(PVTL): \url{https://drive.google.com/drive/folders/1UFcr-b4Ci5BsA72TZWJ77n-J3aneli6l}
\end{itemize}

In terms of Frameworks we have a few options that we
plan to explore:
\begin{itemize}
  \item  JCLAL: \url{https://github.com/ogreyesp/JCLAL}
  \item Libact: \url{https://github.com/ntucllab/libact}
  \item modAL: \url{https://github.com/modAL-python/modAL}
\end{itemize}

In terms of Algorithms, we plan on building an Active Learning related one, 
and we will be using State of the Art methods for mutual comparison.

\section{Experimental Setup}

In terms of Experimental Setup, we want to measure the performance of the 
algorithms (State of the Art vs Active Learning approach) utilizing various 
metrics, on detecting and classifying pedestrian traffic light

\end{document}
